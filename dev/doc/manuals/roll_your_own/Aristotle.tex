% !TEX root = roll_your_own.tex
\chapter{Aristotlean syllogisms}
\label{chap:Aristotle}

Roy Dyckhoff was a fundamental influence on Jape. His program MacLogic \citep{dyckhoff1989malt} guided Bernard and me, both logic novices, through formal tree proofs using logical symbols; after the proof was made it showed us a box-and-line rendering; and it included a Prolog encoding of his intuitionistic decision procedure \citep{dyckhoff1992contraction} which would tell us if we'd made a step such that further progress was impossible. In terms of user interaction we were sure we could do better; in particular we designed Jape so that it didn't guide its users but instead showed them where they were in a proof, and gave them the choice of what to do next. Eventually I even managed to deal with making proofs directly in box-and-line form. But we never managed to incorporate a decision procedure into Jape, and Roy, though he listened politely while we enthusiastically promulgated the advantages of Jape over MacLogic, politely never let on that it was really that decision procedure which was the point of his program. 

Roy's death in 2018 came as a shock. Just after his death an article of his on Aristotlean syllogisms was published \citep{dyckhoff2019indefinite}. I thought it might be fun to be able to check his proofs in Jape (he'd left them as a `logical exercise'). That paper was related to the work of von Plato \citep{von2016aristotle}, who showed how Aristotle was using a deductive logic, so I had to deal with that as well.

It was more difficult than I had expected, and it even needed some changes to Jape, but I was working on it during the spring 2019 Covid lockdown, so I had time. I ended up doing it twice: once with symbolic operators, once again, more recently, with English words. I made the symbolic version first, so I describe it first.

\section{In symbols}

The encodings are in the \textj{examples/Aristotle/in symbols} folder, in \textj{Syllogisms (von Plato).jt}, \textj{Syllogisms (Roy Dyckhoff).jt}.

Von Plato's notation uses four kinds of formula, with terms $P$ and $M$
\begin{quote}
\begin{tabular}{ll}
$∏⁺(P,M)$ & Every $P$ is $M$ \\
$∏⁻(P,M)$ & No $P$ is $M$ \\
$∑⁺(P,M)$ & Some $P$ is $M$ \\
$∑⁻(P,M)$ & Some $P$ is non $M$ \\
\end{tabular}
\end{quote}
\begin{figure}[b]
\centering
\includegraphics[scale=0.5]{pics/Aristotle/square_of_opposition}
\caption{The square of opposition}
\label{fig:Aristotle:square_of_opposition}
\end{figure}
Figure \ref{fig:Aristotle:square_of_opposition} shows the relationships which mediaeval European logicians drew between these formulas: \emph{contradictory} if one must be false when the other is true; \emph{contrary} if they can't both be true (though they can both be false); \emph{subaltern} if one implies the other. Von Plato makes use of subaltern and contradictory, using superscript tack $^{⊥}$ as a `contradictory' operator; I couldn't find a way to make that mark in Unicode, so like Dyckhoff I use $^{*}$. $∏⁺(P,M)^{*}$ is $∑⁻(P,M)$; $∏⁻(P,M)^{*}$ is $∑⁺(P,M)$; and in each case the other way around as well. Dyckhoff points out that $A^{{*}{*}}$ is $A$. There is also a notion of `equivalence' or `reversal': $∏⁻(P,M)$ is equivalent to $∏⁻(M,P)$; $∑⁺(P,M)$ to $∑⁺(M,P)$.

Defining the syntax is almost trivial\footnote{This example has shown me that Jape's syntax-description mechanism is deficient. Nowhere does it say what a syllogism looks like ...}
\begin{japeish}
CLASS VARIABLE P S M X Y               /* TERM, really */\\
CLASS FORMULA A B C\\
\\
PREFIX 100 ∏⁺ ∏⁻ ∑⁺ ∑⁻\\
POSTFIX 50 *\\

SEQUENT IS BAG ⊢ FORMULA
\end{japeish}
\begin{figure}
\centering
\subfigure[with hidden cut and identity]{\centering
\makebox[140pt]{\includegraphics[scale=0.5]{pics/Aristotle/Darapti_1}}\label{fig:Aristotle:Darapti_1}}
\subfigure[with decorated assumption and explicit antecedents]{\centering
\makebox[160pt]{\includegraphics[scale=0.5]{pics/Aristotle/Darapti_2}}\label{fig:Aristotle:Darapti_2}}
\subfigure[without boxes and step names]{\centering
\makebox[155pt]{\includegraphics[scale=0.5]{pics/Aristotle/Darapti_3}}\label{fig:Aristotle:Darapti_3}}
\caption{Aristotelean proof appearances}
\label{fig:Aristotle:Darapti_various}
\end{figure}
%Rules begin fairly normally, except that capriciously I have called tne identity rule `see' (more on this later)
%\begin{japeish}
%INITIALISE seektipselection false       /* yes, we are going to use CUTIN */
%
%RULE see(A) IS INFER A ⊢ A
%
%RULE cut(B) IS FROM B AND B ⊢ C INFER C
%
%STRUCTURERULE IDENTITY   see
%STRUCTURERULE CUT        cut
%
%TACTIC Fail (x) IS SEQ (ALERT x) STOP
%\end{japeish}
The main difficulty of encoding von Plato's logic is the form of linear proofs. Figure \ref{fig:Aristotle:Darapti_1} shows how Jape, without taking account of von Plato's insights, would display a proof of a particular syllogism  -- Darapti: $∏⁺(M,P), ∏⁺(M,S) ⊢ ∑⁺(S,P)$ -- in box style with cuts and identity steps hidden. 

Von Plato tells us that Aristotlean linear proofs list the antecedent formulae, in order, just before each conclusion step. This can require identity (\textj{see}) lines pointing to an earlier derivation of the antecedent; but the first antecedent is allowed to be the last line of a box. In addition, the assumption introduced by the \emph{indirect proof} rule IP has to be specially labelled, as is line 5 in figure \ref{fig:Aristotle:Darapti_2}. Finally, the display should be linear without boxes and even without step names, as in figure \ref{fig:Aristotle:Darapti_3}. The encoding in \textj{Syllogisms (von Plato).jt} is set up to display proofs following these principles, but there's a menu `Appearance' which lets the user adjust various settings.

The Appearance menu makes use of environment variables \textj{priorAntes} (set true), \textj{innerboxes} (set false) and \textj{hidewhy} (set true), plus a special \textj{LAYOUT} mechanism described below. In addition, the tactics used to apply rules make heavy use of \textj{CUTIN} and \textj{see} to ensure that antecedents are, as far as possible, closed with an identity step -- helpful to the \textj{priorAntes} mechanism.

There is a condition on the IP rule (Indirect Proof) which required me to invent a new proviso. Essentially the rule is
\begin{japeish}
FROM A* ⊢ B AND B* INFER A
\end{japeish}
-- if the contrary of $A$ proves $B$ and we can separately prove the contrary of $B$, then by \emph{reductio ad absurdum} we can infer $A$. But the assumption $A^{*}$ has to be relevant to the proof: we can't trivially discharge it (e.g. $B$ cannot be $A^{*}$) and we can't have a proof of $B$ which doesn't make use of $A^{*}$ at all. Von Plato says that $A^{*}$ should have a single discharge: from reading his paper and Dyckhoff's I \emph{think} he means that it is discharged only in the left antecedent arm.\footnote{Because his linear proofs have no means of indicating the scope of an assumption, some confusion is to be expected.} The proviso is currently called \textj{NTDISCHARGE} (for Non-Trivial Discharge), and to avoid too many contrary markers in the proof I made an \textj{ALT} of different versions of \textj{IP}.
\begin{japeish}
RULES IP(B) ARE \\
\tab\tab WHERE NTDISCHARGE ∏⁺(S,P) FROM ∏⁺(S,P) ⊢ B AND B* INFER ∑⁻(S,P) \\ 
 AND WHERE NTDISCHARGE ∏⁻(S,P) FROM ∏⁻(S,P) ⊢ B AND B* INFER ∑⁺(S,P)  \\
 AND WHERE NTDISCHARGE ∑⁺(S,P) FROM ∑⁺(S,P) ⊢ B AND B* INFER ∏⁻(S,P)  \\
 AND WHERE NTDISCHARGE ∑⁻(S,P) FROM ∑⁻(S,P) ⊢ B AND B* INFER ∏⁺(S,P) \\
END
\end{japeish}
\textj{IP} is applied via a tactic in the Rules menu
\begin{japeish}
WHEN (LETGOAL \_A (LAYOUT "IP" ALL IP (LAYOUT ASSUMPTION ("contradicting \%t",\_A)))) \\
\tab\tab\tab                  (ALERT "Please select a consequent to be proved by IP") \\
\end{japeish}
The rule which is applied is either \textj{IP'0}, \textj{IP'1}, \textj{IP'2} or \textj{IP'3}, depending on the conclusion of the step, but the first \textj{LAYOUT} names the step \textj{IP}. The second step applies \textj{LAYOUT} to the first antedecent, giving the text to be appended to `assumption'.

But there's still a problem. The contrary marker $^{*}$ isn't really part of the logic, but it appears in partial proofs when you use the IP rule. Here, for example, is the first step in the proof of Darapti:
\begin{quote}
\includegraphics[scale=0.5]{pics/Aristotle/Darapti_0}
\end{quote}
If the proof develops so that we can determine \textj{\_B} on line 4 then the automatic application of a special rule will make the asterisk disappear:
\begin{japeish}
RULES contra ARE \\
\tab\tab\tab   FROM ∏⁺(S,P) INFER ∑⁻(S,P)*  \\
\tab AND FROM ∏⁻(S,P) INFER ∑⁺(S,P)*    \\  
\tab AND FROM ∑⁺(S,P) INFER ∏⁻(S,P)*  \\
\tab AND FROM ∑⁻(S,P) INFER ∏⁺(S,P)*  \\
END\\
\\
TACTIC "contra-tac" IS LAYOUT HIDEROOT contra\\
\\
AUTOMATCH "contra-tac"
\end{japeish}
\begin{figure}
\centering
\subfigure[first stage of proof]{\centering
\makebox[170pt]{\includegraphics[scale=0.4]{pics/Aristotle/subaltern_0}}\label{fig:Aristotle:subaltern_0}}
\subfigure[after Given is applied to line 3]{\centering
\makebox[170pt]{\includegraphics[scale=0.4]{pics/Aristotle/subaltern_1}}\label{fig:Aristotle:subaltern_1}}
\caption{Proving a derived rule}
\label{fig:Aristotle:subaltern_various}
\end{figure}
Indeed, that's most often what happens. But not always: sometimes we realise what \textj{\_B*} should be, and we want to use \textj{see}, or a Given formula, to determine it. Figure \ref{fig:Aristotle:subaltern_0}, for example, is the first stage of the proof of a derived subaltern rule. There is absolutely nothing that \textj{\_B*} on line 3 can be, other than the Given \textj{∏⁻(P,S)}. We don't want to have to calculate what \textj{\_B} must be: surely the encoding of the theory should do that. 

There's a rule \textj{dcontra}, a tactic \textj{remstar} that uses it, and a tactic \textj{dogiven} that uses \textj{remstar}. The tactic \textj{dogiven(0)} is applied by selecting the formula on line 2 and clicking \textj{∏⁻(P,S)} in the Given menu:
\begin{japeish}
INITIALISE givenMenuTactic dogiven\\
\\
TACTIC dogiven(i) IS SEQ remstar (GIVEN i) \\
\\
TACTIC remstar IS \\
\tab  WHEN (LETGOAL (\_A*) (LAYOUT HIDEROOT dcontra))\\
\tab\tab\tab\tab       SKIP\\
\\       
RULE dcontra IS FROM A INFER A**
\end{japeish}
\textj{remstar} either applies \textj{dcontra} or \textj{SKIP}. Applied to the goal \textj{\_B*}, the \textj{LETGOAL} guard succeeds, and it applies \textj{dcontra}; the rule unifies \textj{\_B*} with \textj{\_A1**} -- i.e. it unifies \textj{\_B} with \textj{\_A1*}, producing the antecedent goal \textj{\_A1}. Then \textj{GIVEN 0} unifies \textj{\_A1} with \textj{∏⁻(P,S)}. Since \textj{\_B} is now \textj{∏⁻(P,S)*}, \textj{contra-tac} tidies up, and we have the proof state in figure \ref{fig:Aristotle:subaltern_1}.

But there's a subtlety. Suppose we select \textj{\_B} on line 2 of figure \ref{fig:Aristotle:subaltern_0} and click \textj{∏⁻(P,S)} in the Given menu. That would apply \textj{dogiven 0}, which would apply \textj{remstar}. Should \textj{LETGOAL (\_A*)} succeed? Certainly \textj{\_B} and \textj{\_A*} unify, but only by changing the proof, replacing \textj{\_B} by \textj{\_A*} and \textj{\_B*} by \textj{\_A**}. That seems to me to be an error: asking whether a proof element has a particular shape should not alter that element. Applying a rule can change the tree, but asking a question should not. So \textj{LETGOAL (\_A*)} will fail on line 2, \textj{remstar} will do nothing, and \textj{GIVEN 0} will unify \textj{\_B} with \textj{∏⁻(P,S)}, as it should (though it's actually the wrong move to make at that point in the proof).

The rest of the encoding is fairly straightforward, though it makes a lot of use of \textj{CUTIN} and \textj{see}, to make the \textj{priorAntes} mechanism work.

\section{In words}

Same filenames -- \textj{Syllogisms (von Plato).jt} and \textj{Syllogisms (Roy Dyckhoff).jt} -- but in \textj{examples/Aristotle/in words}

I thought I might show my grandson the Aristotlean encoding, but I realised that the operators might be a barrier. So I redid it in English. Von Plato says he did it in English first, and used symbols to save space in the published paper. I haven't shown it to my grandson yet, but the proofs do feel a bit easier in English ... hmm: I wonder. That's enough of that.

