% !TEX root = roll_your_own.tex
\chapter{Disproof with Kripke trees} 
\label{chap:Kripke}

I have only one explanation: I needed disproof, it had to be integrated with Jape, it had to be Kripke trees, it was fun doing it.

There's a file \textj{I2L\_disproof.j}, containing
\begin{japeish}
SEMANTICTURNSTILE ⊢ IS ⊧ \\
\\
FORCEDEF ⊤   IS ALWAYS /* top everywhere */ \\
FORCEDEF ⊥   IS NEVER  /* bottom nowhere */ \\
\\
FORCEDEF A∧B IS BOTH (FORCE A) (FORCE B) \\
FORCEDEF A∨B IS EITHER (FORCE A) (FORCE B) \\
FORCEDEF A→B IS EVERYWHERE (IF (FORCE A) (FORCE B)) \\
FORCEDEF ¬A  IS NOWHERE (FORCE A) \\
\\
FORCEDEF ∀x.P(x) IS EVERYWHERE (ALL (actual i) (FORCE (P(i)))) \\
FORCEDEF ∃x.P(x) IS SOME (actual i) (FORCE (P(i)))
\end{japeish}
These forcing definitions are interpreted by Jape, and they give you the capabilities described in the \textj{disproof\_howto} manual, distributed with modern releases.