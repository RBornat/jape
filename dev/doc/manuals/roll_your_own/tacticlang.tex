% !TEX root = roll_your_own.tex
\chapter{The tactic language}
\label{appx:tacticlang}


There are no reserved words in the tactic language. It is written in a very restricted sub-dialect of the formula language, without the restriction that the class of every identifier must be pre-declared. When it comes to the application of a rule --- the simplest kind of tactic --- then the arguments must be stated in the formula language.


Although there aren't any reserved words, there are a lot of tactic language verbs. As in the paragraph language, these are all in upper case. You don't have to avoid these names in the statement of rules and theorems, but if you start using them as tactic parameter names you might confuse things.


Since version 3.0, tactic applications are written in `curried' style: \textit{verb} \textit{arg1}... \textit{argN}, where each argument is bracketed if necessary\footnote{The older `uncurried' style, with arguments provided as a bracketed tuple, is now withdrawn. That means we have curried applications and uncurried definitions. One day...}. If a tactic starts with a verb which isn't one of those listed below, it is treated as an application of a named tactic, rule or conjecture. The verb and arguments of a tactic application are evaluated in the current environment --- that means that any names they contain which are parameters of the current tactic, or parameters of the current \textsc{let...} tactical, are replaced by the corresponding formulae.


When a named tactic is applied to arguments, a new environment is created by zipping together tactic parameters and supplied arguments. If there are too few arguments, the remaining parameters are ignored. When a name is evaluated which doesn't have a value in the current environment, the name itself is taken as the result.


\section{Tactic verbs}


\begin{description}
\item[\textsc{alt} \textit{tactic}... \textit{tactic}]: try each of the tactics in turn, until one is found that succeeds. If none succeeds, \textsc{alt} fails.

\item [\textsc{any} \textit{tactic}]: if \textit{tactic} applies a rule which can succeed in more than one way, select one of them without asking the user. Used when replaying proofs and in some automated proof steps (usually in conjunction with \textsc{match hyp}).

\item [\textsc{applyorresolve} \textit{tactic}]: rules applied by \textit{tactic} will be tried first normally, where both hypotheses and conclusions must unify, and then `by resolution' where only conclusions need unify and extra antecedents are inserted to prove each of the hypotheses. The `resolution' step requires that the logic have a \textsc{cut} structure rule.


\item [\textsc{assign} \textit{variable} \textit{value}]: the named variable, which must be part of the global enviroment (see \appxref{GUIlang}) is assigned the given value. Some variables can't be altered once anything has been loaded into the tactic/rule/conjecture store\footnote{Those variables are properly parameters and for clarity we ought to have a syntax for handling them. Patience, patience.}.


\item [\textsc{do} \textit{tactic}]: apply \textit{tactic} repeatedly until it fails, then \textsc{do} succeeds.


\item [\textsc{evaluate} \textit{formula} ]: evaluate one of a fixed number of built-in judgements. Where used, this tactic is explained in one of the distributed encodings; at the time of writing it is used only in the functional programming encoding to evaluate \textsc{assoceq}(\textit{f1}, \textit{f2}), a judgement that two formulae are identical when rewritten with maximal use of associativity laws. \textsc{evaluate} is intended to be the basis, one day, for a mechanism of communication with oracle programs.


\item [\textsc{explicit} \textit{name},..., \textit{name}]: succeeds if every \textit{name} is a parameter for which an argument has been supplied. I think. Opposite of \textsc{implicit} below. I think.


\item [\textsc{flatten} \textit{formula}]: `flattens out' all subformulae of \textit{formula} in the conclusion of the current problem sequent by rewriting according to the rules of associativity. It's based on the same machinery as \textsc{assoceq}; see \textsc{evaluate} above and \chapref{funcprog}.


\item [\textsc{fold} \textit{rulename} \textit{tactic}]: Automatically `folds' collections of rules. See \chapref{funcprog} above.
\item [\textsc{foldhyp} \textit{pattern tactic}]: Automatically `folds' hypotheses. See \chapref{funcprog} above.

\item [\textsc{given} \textit{n}]: apply the \textit{n}th given formula to close a tip.

\item [\textsc{if} \textit{tactic}]: run \textit{tactic}, but succeed even if it fails.


\item [\textsc{implicit} \textit{name}... \textit{nameN}]: succeeds if none of of \textit{name}1... \textit{nameN} is a parameter for which an argument has been supplied. I think. Opposite of \textsc{explicit} above. I think.


\item [\textsc{jape} \textit{stuff} ]: probably deserves a section on its own. Was originally called the `AdHoc' tactic, and it shows. Usually \textit{stuff} is nothing more than ``fail \textit{message}'', but can also be ``showalert \textit{message}'' and ``write \textit{message}'' and lots more which it would be tedious and embarrassing to list. Likely to change frequently and without notice.


\item [\textsc{layout} \textit{fmt} \textit{tactic} ... \textit{tactic}]: a tactical which can hide part of a proof. It has a baroque syntax. \textit{fmt} can be 
\begin{itemize}
\item \textsc{hideroot}
\item \textsc{hidecut} 
\item \textsc{compress}
\item \textsc{compress} \textit{label} \textit{antes}
\item \textit{label} \textit{antes}
\end{itemize}
in which \textit{label} can be \begin{itemize}
\item a string
\item an identifier
\item an unknown
\item a number 
\item a non-empty bracketed tuple of the above
\end{itemize}
and \textit{antes} is a possibly-empty bracketed tuple of numbers or the word \textsc{all}. It's possible to omit \textit{antes}, in which case \textsc{all} is assumed; \textsc{compress} on its own means \textsc{compress} "\%s" \textsc{all}.

In every case the tactics are run to produce a tree, part of which is then shown. \textsc{hideroot} shows the tree without its root (and, I think, works properly if there are several antecedents); \textsc{hidecut} hides the root iff it's a \textsc{cut} step; the others show the tree with some of its antecedents hidden. Antecedents are numbered from 0, so if \textit{antes} is (1,3), for example, only the second and fourth antecedents will be shown. By double-clicking on the justification produced by \textsc{layout} it's possible to toggle between various different displays of the tree -- essentially between `hidden' and `full' form. There are various variables, with names ending with `fmt' (see variable table below) which affect what happens with this tactical. (I wish I'd documented the thing when I still understood it, and I promise to update this if and when I ever understand it again.)

\item [Various \textsc{let}... \textit{tactic... tactic} tacticals]: tactics which test some property: if they succeed, bind some names or some unknowns and proceed with \textit{tactic... tactic}; otherwise they fail. Then can be used as be used as `guarded tactics' which can be used in \textsc{when}. Different tacticals use different mechanisms to make the test and the bindings. 

\begin{description}
\item [] The first group simply binds something to a name (may be an identifier or an unknown)
\item [\textsc{letargtext} \textit{name tactic... tactic}]: if the user has made a \textit{single} text selection, bind the text to \textit{name} and succeed; 
\item [\textsc{letgoalpath}  \textit{name tactic... tactic}]: bind the path to the current goal with \textit{name}, and succeed. 
\end{description}

\begin{description}
\item[] The second group unify a pattern with something from the proof tree, but the unification is asymmetric: if will fail if it causes a change in the proof tree -- i.e. you can bind unknowns in the pattern but not in the tree.

\item [\textsc{letargsel} \textit{pattern tactic... tactic}]: if the user has made a \textit{single} text selection, parse that text and match it with \textit{pattern}; 
\item [\textsc{letconc} \textit{pattern tactic... tactic}]: if the user has formula-selected a \textit{single} conclusion, match it with \textit{pattern}; 

\item [\textsc{letconcfind} \textit{pattern tactic... tactic}]: if the user has made a single text selection \textit{f} in a conclusion formula \textit{C}, so that \textit{C} consists of \textit{f1} followed by \textit{f} followed by \textit{f2}, and the text \textit{f1} (\textit{f}) \textit{f2} is a parseable formula, and the formula (\textit{C}, \textit{f1} (\textit{f}) \textit{f2}) matches \textit{pattern}, then: if \textit{C} is not structurally the same formula as \textit{f1} (\textit{f}) \textit{f2} succeed; if it is the same, succeed silently, without running the sequence \textit{tactic... tactic}. (wow!)

\item [\textsc{letconcsubstsel} \textit{pattern tactic... tactic}]: like \textsc{letsubstsel} (q.v. below) except that the text-selection must be made in a conclusion (right-hand side) formula.

\item [\textsc{letgoal} \textit{pattern tactic... tactic}]: if the current problem sequent has a single conclusion formula, match it with \textit{pattern}; 

\item [\textsc{lethyp} \textit{pattern tactic... tactic}]: if the user has formula-selected a single hypothesis formula, match it with \textit{pattern}; 
\item [\textsc{lethyp2} \textit{pattern1 pattern2 tactic... tactic}]: If the user has formula-selected two hypothesis formulae, match them with \textit{pattern1} and \textit{pattern2} (order of selection irrelevant -- Jape tries both ways if necessary); 
\item [\textsc{lethyps} \textit{pattern tactic... tactic}]: if the user has formula-selected one or more hypothesis formulae, match them, as a tuple, with \textit{pattern}; 


\item [\textsc{lethypfind} \textit{pattern tactic... tactic}]: just like \textsc{letconcfind} (q.v. above), except that the single text-selection must be made in a hypothesis formula;

\item [\textsc{lethypsubstsel} \textit{pattern tactic... tactic}]: like \textsc{letsubstsel} (q.v. below) except that the text-selection must be made in a hypothesis formula of the current problem sequent;

\item[\textsc{letlhs, letrhs}]: ??;

\item [\textsc{letmultisel} \textit{pattern tactic... tactic}]: ; match all the user's text-selections, expressed as a tuple of formulae, with \textit{pattern}; ;
\item [\textsc{letoccurs}]: ??;

\item [\textsc{letsubstsel} \textit{pattern tactic... tactic}]: if the user has made a number of text selections within a single formula, each an instance of an identical sub-formula, convert the formula to a substitution (see chapter 1) and match it with \textit{pattern}; 

\end{description}

\begin{description}
\item[] Last of all \textsc{letunify} uses symmetrical unification
\item [\textsc{letunify} \textit{pat1 pat2} \textit{tactic... tactic}]: unify \textit{pat1} with \textit{pat2};
\end{description}


\item [\textsc{mapterms} \textit{tactic}]: if the current problem sequent has a conclusion which is a single formula, try to apply \textit{tactic} (which is probably some sort of rewrite rule) to each of the structural subformulae of that conclusion formula.

\item [\textsc{match} \textit{tactic}]: runs \textit{tactic} so that any rules which it applies are required to succeed without visibly changing the unification context --- that is, without changing the interpretation of any unknowns in the problem sequent.

\item [\textsc{prove} \textit{tactic}]: detaches the current goal from the proof tree; tries to prove it; and then plugs in the proof if it's complete, otherwise fails. A way of ensuring that a tactic builds a subtree with no open tips.

\item [\textsc{replay} \textit{tactic}]: run \textit{tactic} but use term equality (up to \ensuremath{\alpha}-conversion and elimination of substitutions) instead of unification. Used in proof loading, because it seems to be a bit faster than the normal mechanism.

\item [\textsc{resolve} \textit{tactic}]: rules and theorems applied by \textit{tactic} will all be applied `by resolution' in which only the conclusions need unify and extra antecedents are inserted for each hypothesis. See \textsc{simpleapply} below and \textsc{applyorresolve} above.


\item [\textsc{sameprovisos} \textit{tactic}]: rules applied by \textit{tactic} mustn't add or delete any provisos from the current unification context. Used to be used in proof reloading; may now be obsolete.


\item [\textsc{seq} \textit{tactic}... \textit{tacticN}]: run the tactics in sequence. Fail if any of them fails.


\item [\textsc{simpleapply} \textit{tactic}]: each of the rules applied by \textit{tactic} will be applied in `normal' style, without using the `by resolution' mechanism.


\item [\textsc{skip}]: succeed.


\item [\textsc{theoryalt}]: generated internally, and used when a particular mechanism, used only in the functional programming encoding, is searching for and caching rules. It's all rather horrid and extremely \textit{ad hoc}, and no further details will ever be released.


\item [\textsc{unfold} \textit{rulename tactic}]: see \textsc{fold} above.
\item [\textsc{unfoldhyp} \textit{formula tactic}]: see \textsc{foldhyp} above.


\item [\textsc{unique} \textit{tactic}]: run \textit{tactic} so that any rules it applies are required to succeed in only one way (i.e. prevents application of those rules from offering the user a choice of alternative matches). Used in proof reloading.


\item [\textsc{when} \textit{guardedtactic}.... \textit{guardedtactic tactic}]: try each of the guarded tactics in turn until one is found whose guard unifies, then run the tactics it guards; if none of the guards succeeds, run the final alternative \textit{tactic}. The guarded tactics must each be one of the \textsc{let...} variety: see \textsc{LET...} tacticals above, and `guarded and binding tactics' below.


\item [\textsc{withargsel} \textit{tactic}]: run \textit{tactic}, giving it as argument the current text-selection, provided that there is only a single text selection and it parses properly as a formula. Fails if there is a single text selection but it can't be parsed.


\item [\textsc{withconcsel} \textit{tactic}]: if the user has formula-selected a conclusion formula, rules applied by \textit{tactic} must consume it (that is, explicitly unify with it).


\item [\textsc{withcontinuation} \textit{tactic1 tactic tactic...}]: \textit{tactic1} is run so that its continuation is the sequence \textit{tactic tactic...}. Has no effect unless \textit{tactic1} ends with an \textsc{alt/theoryalt}; then it will ensure that no alternative of that \textsc{alt} succeeds unless the sequence \textit{tactic tactic...} succeeds afterwards. Makes alternative choice a little more lazy.


\item [\textsc{withformsel} \textit{tactic}]: a combination of \textsc{withconcsel} above and \textsc{withhypsel} below.


\item [\textsc{withhypsel} \textit{tactic}]: if the user has formula-selected a hypothesis formula, rules applied by \textit{tactic} must consume it (that is, explicitly unify with it).


\item [\textsc{withselections} \textit{tactic}]: a combination of \textsc{withargsel}, \textsc{withconcsel} and \textsc{withhypsel}.


\item [\textsc{withsubstsel} \textit{tactic}]: normally used inside a \textsc{letsubstsel} tactical. The user's text selections have to be entirely within one of the hypotheses or conclusions of the current problem sequent: rewrite that hypothesis or conclusion as a substitution form, based on the text selections given, and then run \textit{tactic}. Fails noisily if the text selections don't describe a substitution in just the right way; fails normally if the substitution is described, but \textit{tactic} fails. Rules applied by \textit{tactic} must consume (i.e. explicitly unify with) the reconstructed formula.


In addition to all those there are two that can occur inside a \textit{formula} inside a tactic:


\item [\textsc{antiquote} ( \textit{formula} )]: everything inside \textit{formula} is liable to `evaluation' in the current tactic environment, unless it is \textsc{quote}d. Arguments in applications of tactics are treated as if they were \textsc{antiquote}d.


\item [\textsc{quote} ( \textit{formula} )]: nothing inside \textit{formula} is liable to `evaluation' unless it is \textsc{antiquote}d.

\end{description}

\section{The `current problem sequent', the `goal' and the `target'}


When you start a proof there is only one problem sequent. When you apply a rule with two antecedents, there are two to choose from. When you are well into a proof, there may be many.


Each time Jape makes a proof step (by application of a rule or in a small number of other ways, mostly to do with the more exotic of the tactics like \textsc{find} or \textsc{flatten}, and sometimes caused by the dialogue language) it selects a new problem sequent if the current one is closed, or replaced by a subtree. It always finds the `next rightmost unclosed tip' and makes that the current problem sequent. The `next rightmost unclosed tip' is the first one in the fringe of the tree to the right of the current one, or the first one in the fringe if there isn't one to the right of the current one.


The current problem sequent is called the `goal'; the problem sequent from which we moved to the current one because application of a rule succeeded is called the `target' (not a very good name, `target', but that's the way it is).


\section{Guarded and binding tactics}


Jape's tactic language is `eager' --- whether it should be so continues to be a matter of debate --- with backtracking on failure. If a tactic fails, then the enclosing tactic either fails, or if it is an \textsc{alt}, tries another alternative starting from the state in which it first applied the sub-tactic that failed. That sort of backtracking search is fine sometimes, but not always. It can be modified --- slightly --- by \textsc{withcontinuation}.


The \textsc{when} tactical takes `guarded tactics' and applies them carefully, accepting the result of the first one of them whose guard succeeds. Note that the whole guarded tactic may fail after its guard has succeeded, and in that case \textsc{when} won't backtrack, it will simply fail.


Each of the guarded tacticals --- they are all called \textsc{let...} --- takes a \textit{pattern} and a \textit{tactic} sequence. The \textit{pattern} is `matched' with something -- i.e. unified with the condition that \emph{the proof tree isn't changed}. If matching succeeds then the environment is updated to reflect the matching. Roughly speaking you can assume, when matching, that unknowns in \textit{pattern} will be added to the environment as parameters corresponding to the stuff they unified with, and if they are used again in the tactic sequence, they will be replaced by that same stuff. You don't have to worry that the unknowns you use might already appear in the unification context: Jape invents new ones so that the effects of a successful binding never leak into the unification context used in proof steps (though you can force it to using \textsc{unify}, if you wish).

 