\chapter{Encoding Hoare logic}
\label{chap:Hoare}

This was always the target. I haven't solved the obvious arithmetic problem, nor even linked up to a finite arithmetic oracle (help welcome!), but I went some way to make it easy to use. It's in \textj{examples/hoare\_logic} in \textj{hoare.jt}. It's closely tied to the treatment in my book (Proof and Disproof in Formal Logic, OUP, to be published August 2005).

\section{Syntax (\textsc{hoare\_syntax.j})}

Variables can start with any letter. Formula names start with any letter except K. Constant names start with K.
\begin{japeish}
CLASS VARIABLE a b c d e f g h i j k l m n o p q r s t u v w x y z  \\
CLASS FORMULA A B C D E F G H I J L M N O P Q R S T U V W X Y Z \\
CLASS CONSTANT K \\
CLASS BAG FORMULA Γ 
\end{japeish}

\textit{mod} is used in various examples. \textit{length} is used when dealing with arrays.
\begin{japeish}
CONSTANT mod length \\
CONSTANT ⊥ ⊤
\end{japeish}

\textj{simplifiesto}, \textj{equivto}, \textj{conjoins} and \textj{dependson} are operators used in the simplification of array-assignment preconditions (\secref{Hoare:arrayassignment}). The treatment of quantifiers comes from the natural deduction encoding of \chapref{I2L}, but`integer is used in place of actual; I have to define actual as well because of the requirements of \textsc{pushsyntax} (\secref{Hoare:pushsyntax}). \textj{computes} and \textj{defines} are used in the treatment of definedness (\secref{definedness}).
\begin{japeish}
INFIX   5L  ≜ /* equals def */ \\
INFIX   5L  simplifiesto equivto conjoins dependson /* see hoare\_arith.j */ \\
 \\
PREFIX  10  actual integer /* actual not used, but we have to satisfy PUSHSYNTAX */ \\
POSTFIX 10  computes defined \\
\end{japeish}
 
Semicolon and becomes are essential operators.
\begin{japeish}
INFIX 10 L ; \\
INFIX 12 L :=
\end{japeish}

Circleplus and mapsto are used in the treatment of array assignment (\secref{I2L:arrayassignment}).
\begin{japeish}
INFIX 50 L ⊕ /* circle plus */ \\
INFIX 60 L ↦ /* maps to */
\end{japeish}

The infix connectives and leftfix quantifiers:
\begin{japeish}
INFIX   100R    → ↔ /* implies, iff */ \\
INFIX   120L    ∨ \\
INFIX   140L    ∧ \\
LEFTFIX 180 ∀ . \\
LEFTFIX 180 ∃ .
\end{japeish}

Relational and arithmetic operators:
\begin{japeish}
INFIX   300L    <   >   ≤   ≥   ≠   =  \\
INFIX   400 L   + - \\
INFIX   410 L   × ÷ \\
INFIX   420 R   ↑
\end{japeish}

Prefix negation:
\begin{japeish}
PREFIX  1200    ¬
\end{japeish}

Juxtaposition and substitution:
\begin{japeish}
JUXTFIX     9000 \\
SUBSTFIX    10000   « E / x  »
\end{japeish}

How quantifiers bind (only single variables considered):
\begin{japeish}
BIND x SCOPE A IN ∀x . A \\
BIND x SCOPE A IN ∃x . A
\end{japeish}

Sequents (including a turnstile):
\begin{japeish}
SEQUENT IS BAG ⊢ FORMULA
\end{japeish}

Shorthand rule definition variables, very much as usual:
\begin{japeish}
INITIALISE autoAdditiveLeft true /* allow rules to be stated without an explicit left context */ \\
INITIALISE interpretpredicates true
\end{japeish}

Essential bracketing forms:
\begin{japeish}
OUTFIX { } /* for assertions */ \\
OUTFIX [ ] /* for indexing */
\end{japeish}

Instruction forms (I prefer `tilt' to `abort', but I don't think I used it all in the end):
\begin{japeish}
OUTFIX if then else fi \\
OUTFIX while do od \\
CONSTANT skip tilt
\end{japeish}

Transitive reasoning is possible (see \chapref{funcprog} --- but I don't think I exploited it):
\begin{japeish}
INITIALISE hidetransitivity true \\
INITIALISE hidereflexivity true
\end{japeish}

Finally, the menu of difficult-to-type symbols that the GUI shows you when you have to type a formula:
\begin{japeish}
KEYBOARD → ↔ ∧ ∨ ¬ ⊥ ∀ ∃ ⊢ ⊧ ≤ ≥ ≠ ≜ × ÷ ↦ ⊕ « »
\end{japeish}

\subsection{Juxtaposition and Hoare triples}

I'd love to have defined the syntax so that, for example
\begin{japeish}
\{Q\} x:=E \{R\}
\end{japeish}
was parsed properly, but I couldn't. It's parsed, currently, as 
\begin{japeish}
(\{Q\} x):=(E \{R\})
\end{japeish}
--- a juxtaposition of \{Q\} and $x$ on the left of the infix operator :=, a juxtaposition of $E$ and \{R\} on the right! Oh dear.

There is no solution to this problem: it's inherent in Jape's simple-minded bottom-up priority-based parser generator. Put \textsc{justfix} high, and it gets \textj{a[i]:=7} right. Put it low, and it would get \textj{\{Q\} x:=E \{R\}} right. I thought that bracketing array element expressions would be unbearable, so I chose high. You have to bracket single assignments, and sequences, to avoid the consequential problem with triples. Conditionals and loops have built-in brackets, so they are automatically ok.

\subsection{\textsc{pushsyntax}}
\label{Hoare:pushsyntax}

I wanted to use a couple of encodings, or partial encodings, that had been developed before. However these things each have their own syntax, and in particular their own hierarchy of operator priorities. Aligning yourself with one previously-defined hierarchy may be painful, but is possible. Aligning yourself with two may be impossible.

\textj{hoare.jt} contains, therefore, a couple of uses of \textsc{push}/\textsc{popsyntax}. Provided that the `calling' encoding defines all the operators of the `called' encoding, and with the same meaning in each case (e.g. ∧ is \textsc{infix} in both), the ordering of priorities doesn't matter: formulae will be read and printed (outside the \textsc{push}/\textsc{popsyntax} bracketing) using the `calling' theory's definitions.

In loading the I2L encoding, I took the opportunity to change the treatment of quantifiers. Hoare logic quantifies only over integers: \textj{I2L\_integer\_quantifiers.j} replaces `actual' with `integer' and also acknowledges that you can provide a subformula
\section{Rules}

Skip and tilt (tilt isn't in the menus):
\begin{japeish}
RULE "skip" IS \{A\} skip \{A\} \\
RULE "tilt" IS \{⊥\} tilt \{A\}
\end{japeish}
 
Sequence:
\begin{japeish}
RULE sequence(C) IS FROM \{A\}F\{B\} AND  \{B\}G\{C\} INFER  \{A\}(F;G)\{C\}
\end{japeish}

It's very convenient sometimes to be able to insert intermediate assertions into a sequence (a kind of cut). This could be done in various ways, but I did it bluntly, so that I could write tactics to take Ntuples apart no matter how large. It exploits the fact that triples are parsed as juxtapositions (e.g. A\{B\}C\{D\} is parsed as ((A\{B\})C)\{D\}, and A will be something like \{A0\}A1\{A2\}...An):
\begin{japeish}
RULES "Ntuple" ARE \\
\tab FROM A\{B\} AND \{B\}C\{D\} INFER A\{B\}C\{D\} \\
AND FROM \{A\}B\{C\} AND \{C\}D\{E\} INFER \{A\}(B\{C\}D)\{E\} \\
END
\end{japeish}

Consequence is straightforward:
\begin{japeish}
RULE "consequence(L)" IS FROM A→B AND \{B\} F \{C\} INFER \{A\} F \{C\} \\
RULE "consequence(R)" IS FROM \{A\} F \{B\} AND B→C INFER \{A\} F \{C\}
\end{japeish}

After that it gets interesting.

\subsection{Variable assignment and simplifiesto}
\label{sec:Hoare:simplifiesto}

In my book the assignment axiom is $ \{(E \text{ computes}) @ A_{E}^{x}\}\, x:=E \,\{A\}$, where $E \text{ computes}$ is supposed to pick out problems like division / remaindering by zero and array element misaddressing. I wanted to do the same thing in Jape: $x+1 \text{ computes}$, for example, is just $\<true>$, and should vanish from the proof. So I implemented some tactical machinery.

The axiom is replaced by a rule, with the simplification in an antecedent: 
\begin{japeish}
RULE "variable-assignment" IS \\
\tab FROM R«E/x»∧(E computes) simplifiesto Q \\
\tab INFER \{Q\} (x:=E) \{R\}
\end{japeish}
This is a very non-committed rule --- all it insists on is a match for x:=E --- and the action is all in the tactic that's applied to the antecedent. 

The menu entry for assignment is
\begin{japeish}
ENTRY "variable-assignment" IS \\
\tab BackwardOnlyA \\
\tab \tab (QUOTE (\{\_A\} (\_x := \_E) \{\_B\})) \\
\tab \tab (Noarg (perhapsconsequenceL (SEQ "variable-assignment" simpl))) \\
\tab \tab "variable-assignment" \\
\tab \tab "{A}(x:=E){B}"
\end{japeish}
which, when the error-checking is stripped away (see \chapref{I2L} for explanation) is
\begin{japeish}
perhapsconsequenceL (SEQ "variable-assignment" simpl)
\end{japeish}

PerhapsconsequenceL tries to apply its tactic, and if that fails inserts a consequence step:
\begin{japeish}
TACTIC perhapsconsequenceL (tac) IS \\
\tab ALT \\
\tab \tab tac \\
\tab \tab (SEQ "consequence(L)" fstep (trueforward tac))
\end{japeish}
(again, see \chapref{I2L} for explanation of fstep and trueforward --- but I don't recall why it uses trueforward at all). It isn't as wasteful as it looks --- it doesn't run simpl twice, because the rule won't match if you need consequence.

Simpl, then, is the tactic that's applied to the simplifiesto formula in the antecedent. It's a structural recursion on the left-hand-side of the antecedent, which takes care not to unify any unknowns it comes across. The tactic itself is in \textj{hoare\_arith.j}:
\begin{japeish}
TACTIC simpl IS \\
\tab LAYOUT HIDEROOT \\
\tab \tab (ALT \\
\tab \tab \tab (LETGOAL (\_E∧(\_x computes) simplifiesto \_F) \\
\tab \tab \tab \tab (UNIFY \_F \_E) (MATCH "arith\_var")) \\
\tab \tab \tab (LETGOAL (\_E∧(\_K computes) simplifiesto \_F) \\
\tab \tab \tab \tab (UNIFY \_F \_E) (MATCH "arith\_const")) \\
\tab \tab \tab (SEQ (MATCH "arith\_index") simpl equiv equiv) \\
\tab \tab \tab (SEQ (MATCH "arith\_single") simpl) \\
\tab \tab \tab (SEQ (MATCH "arith\_double") simpl simpl) \\
\tab \tab \tab (SEQ (MATCH "arith\_div") simpl simpl equiv))
\end{japeish}
\textsc{hideroot} hides a step and displays only its antecedents. Because it's applied at every simpl step, they all disappear. Because each step is applied with \textsc{match}, it won't alter any unknowns in the tree.

The first alternative eliminates variables; the second eliminates constants; the third deals with array-element formulae; the fourth with prefix and postfix operators; the fifth with infix operators other than div and mod; the last with div and mod.

The rules it uses do the work. Before \textj{arith\_var} and \textj{arith\_const} a \textsc{unify} step is necessary to define the output unknown \_F; without it the following \textsc{match} step would fail.
\begin{japeish}
RULE "arith\_var" IS INFER A∧(x computes) simplifiesto A \\
RULE "arith\_const" IS INFER A∧(K computes) simplifiesto A
\end{japeish}

\textj{Arith\_single} only has to deal with negation:
\begin{japeish}
RULE "arith\_single" IS \\
\tab FROM E∧(A computes) simplifiesto F \\
\tab INFER E∧(¬A computes) simplifiesto F
\end{japeish}

\textj{Arith\_double} deals with all the infix operators, simplifying each side separately:
\begin{japeish}
RULES "arith\_double" ARE \\
\tab FROM E∧(A computes) simplifiesto F \\
\tab AND F∧(B computes) simplifiesto G \\
\tab INFER E∧(A → B computes) simplifiesto G \\
AND \\
\tab FROM E∧(A computes) simplifiesto F \\
\tab AND F∧(B computes) simplifiesto G \\
\tab INFER E∧(A ↔ B computes) simplifiesto G \\
AND \\
... \\
END
\end{japeish}

In effect \textj{arith\_div} converts A ÷ B computes into B≠0, and treats A mod B similarly, but it generates an antecedent G∧B≠0 equivto H, to which simpl applies the equiv tactic (discussed below) in order not to insert too many copies of B≠0.
\begin{japeish}
RULES "arith\_div" ARE \\
\tab FROM E∧(A computes) simplifiesto F \\
\tab AND F∧(B computes) simplifiesto G \\
\tab AND G∧B≠0 equivto H\\
\tab INFER E∧(A ÷ B computes) simplifiesto H \\
AND \\
\tab FROM E∧(A computes) simplifiesto F \\
\tab AND F∧(B computes) simplifiesto G \\
\tab AND G∧B≠0 equivto H\\
\tab INFER E∧(A mod B computes) simplifiesto H
END
\end{japeish}

\textj{Arith\_index} converts a[F] computes into 0≤F∧F<length(a), and generates equivto antecedents that are dealt with by two calls of equiv.
\begin{japeish}
RULE "arith\_index" IS \\
     FROM E∧(F computes) simplifiesto G \\
\tab AND G∧0≤F equivto H \\
\tab AND H∧F<length(a) equivto I\\
\tab INFER E∧(a[F] computes) simplifiesto I 
\end{japeish}

The equiv tactic looks for occurrences of F inside E, and if it finds one, declares E∧F equivto E. Then it replaces ⊤∧E with E (this can happen in the while rule):
\begin{japeish}
TACTIC equiv IS \\
\tab SEQ \\
\tab \tab (LAYOUT HIDEROOT (ALT (SEQ "arith\_dup" conjoinstac) SKIP)) \\
\tab \tab (LAYOUT HIDEROOT (ALT "true\_equiv" "equiv\_default"))
\end{japeish}

\textj{Conjoinstac} is another structural recursion, but only dealing with conjunctions:
\begin{japeish}
TACTIC conjoinstac IS \\
\tab LAYOUT HIDEROOT\\
\tab \tab (ALT \\
\tab \tab \tab (MATCH "arith\_conjoins0")\\
\tab \tab \tab (SEQ (MATCH "arith\_conjoinsL") conjoinstac)\\
\tab \tab \tab (SEQ (MATCH "arith\_conjoinsR") conjoinstac))
\end{japeish}
\begin{japeish}
RULES "arith\_conjoins0" ARE \\
\tab INFER E∧F conjoins F \\
AND INFER F conjoins F \\
END\\
RULE "arith\_conjoinsL" IS \\
\tab FROM E conjoins G INFER E∧F conjoins G \\
RULE "arith\_conjoinsR" IS \\
\tab FROM F conjoins G INFER E∧F conjoins G
\end{japeish}

It's a lot of work, but Jape gets through it (slowly: you may notice a pause when you apply the assignment step; this tactic is the main reason why the Jape engine is now compiled rather than interpreted by OCaml) and it means that the definedness issues of assignment can be addressed.

\subsection{Choice}

Choice (if-then-else-fi) is dealt with in much the same way as variable assignment. The rule is
\begin{japeish}
RULE "choice" IS \\
\tab FROM (E→A)∧(¬E→B)∧(E computes) simplifiesto G  \\
\tab AND \{A\} F1 \{C\}  \\
\tab AND \{B\} F2 \{C\}  \\
\tab INFER \{G\} if E then F1 else F2 fi \{C\}
\end{japeish}
and the menu entry boils down to
\begin{japeish}
perhapsconsequenceL (SEQ "choice" simpl fstep fstep)
\end{japeish}

\subsection{Array-element assignment}

Array-element assignment is similar to variable assignment, but it generates a more intricate treatment of computes:
\begin{japeish}
RULE "array-element-assignment" IS  \\
\tab FROM B«a⊕E↦F/a»∧(a[E] computes) simplifiesto C  \\
\tab AND C∧(F computes) simplifiesto D \\
\tab INFER \{D\}(a[E]:=F)\{B\}
\end{japeish}
The menu tactic, shorn of its error checking, is
\begin{japeish}
perhapsconsequenceL \\
\tab (QUOTE \\
\tab \tab (LETGOAL (\{\_A\} (\_a[\_E] := \_F) \{\_B\}) \\
\tab \tab \tab "array-element-assignment" \\
\tab \tab \tab ("length\_simpl" \_a \_E \_F) \\
\tab \tab \tab simpl \\
\tab \tab \tab simpl))
\end{japeish}
Length\_simpl replaces formulae of the form length(a⊕E↦F) with length(a) (they are equivalent by definition, and visual clutter without the simplification). It's done with a straightforward rewrite rule:
\begin{japeish}
TACTIC "length\_simpl"(a,E,F) IS \\
\tab LAYOUT HIDEROOT ("arith\_length" a E F) \\
RULE "arith\_length"(a,E,F,OBJECT x) IS \\
\tab FROM A«length(a)/x» simplifiesto B \\
\tab INFER A«length(a⊕E↦F)/x» simplifiesto B
\end{japeish}

\subsection{While}

While loops are dealt with in the same sort of way:
\begin{japeish}
RULE "while"(I, M, OBJECT Km) WHERE FRESH Km IS \\
\tab FROM ⊤∧(E computes) simplifiesto G \\
\tab AND I→G  \\
\tab AND \{I∧E\}F\{I\} \\
\tab AND I∧E→M>0 \\
\tab AND integer Km  ⊢ \{I∧E∧M=Km\}F\{M<Km\} \\
\tab INFER \{I\}while E do F od\{I∧¬E\}
\end{japeish}
The menu tactic boils down to
\begin{japeish}
perhapsconsequenceLR (SEQ "while" simpl maybetrueimpl fstep fstep)
\end{japeish}
The last antecedent isn't fstepped, because it's hypothetical (I don't know how to integrate it into the proof properly: there's no hyp for boxes; can you see how to do it?). Maybetrueimpl hides I->G in case G is ⊤. (proving it with the rules, no cheating!):
\begin{japeish}
TACTIC maybetrueimpl IS \\
\tab ALT \\
\tab \tab (SEQ (LAYOUT HIDEROOT) (MATCH "→ intro") (LAYOUT HIDEROOT) (MATCH "truth")) \\
\tab \tab SKIP
\end{japeish}

\section{An arithmetic sledgehammer}

Arithmetic and natural deduction are what you actually do in a Hoare-logic proof: the program-logic steps are trivial. The ND steps, and the substitutions in program logic produce situations like $\Gamma, a[i]=2 |- a[i]+1=3$, which is so obviously true. Jape might expect you to prove it in Peano arithmetic, but life's too short; it might hitch up with a finite arithmetic oracle, but I haven't seen a way past licensing problems with any of those and therefore haven't seriously investigated the possibility.

So I borrowed an idea of Krysia Broda's, and implemented an `obviously' step, to be used to prove obvious arithmetic truths. Of course it can be misused, and of course it makes the whole encoding completely unsound, but that has to be ok in this case. The rules are
\begin{japeish}
RULE "obviously0" IS INFER A  \\
RULE "obviously1" IS FROM A INFER B  \\
RULE "obviously2" IS FROM A AND B INFER C  \\
RULE "obviously3" IS FROM A AND B AND C INFER D  \\
RULE "obviously4" IS FROM A AND B AND C AND D INFER E 
\end{japeish}
and the tactic which applies them is
\begin{japeish}
TACTIC obviouslytac IS \\
\tab WHEN \\
\tab (LETHYPS \_A /* 1 or more selections, as a tuple */ \\
\tab \tab (LETLISTMATCH \_A1 \_B \_A /* \_A=(\_A1, ...), \_B=(...) */ \\
\tab \tab \tab (WHEN \\
\tab \tab \tab \tab (LETLISTMATCH \_B1 \_C \_B /* \_B=(\_B1,...), \_C=(...) */ \\
\tab \tab \tab \tab \tab (WHEN \\
\tab \tab \tab \tab \tab \tab (LETLISTMATCH \_C1 \_D \_C ... etc. ...)\\
\tab \tab \tab \tab \tab \tab (LAYOUT "obviously, from" ALL \\
\tab \tab \tab \tab \tab \tab \tab "obviously2" (WITHHYPSEL (hyp \_A1)) (WITHHYPSEL (hyp \_B1))))) \\
\tab \tab \tab \tab (LAYOUT "obviously, from" ALL \\
\tab \tab \tab \tab \tab "obviously1" (WITHHYPSEL (hyp \_A1)))))) \\
\tab (LAYOUT "obviously" ALL "obviously0")
\end{japeish}

\section{N-ary intro and elim steps}

If you have a conclusion which is a long conjunction, it takes lots of ∧ intro steps to break it part, and 
Arithmetic and natural deduction are what you actually do in a Hoare-logic proof: the program-logic steps are trivial. The ND steps, and the substitutions in program logic produce situations like $\Gamma, a[i]=2 |- a[i]+1=3$, which is so obviously true. Jape might expect you to prove it in Peano arithmetic, but life's too short; it might hitch up with a finite arithmetic oracle, but I haven't seen a way past licensing problems with any of those and therefore haven't seriously investigated the possibility.

So I borrowed an idea of Krysia Broda's, and implemented an `obviously' step, to be used to prove obvious arithmetic truths. Of course it can be misused, and of course it makes the whole encoding completely unsound, but that has to be ok in this case. The rules are
\begin{japeish}
RULE "obviously0" IS INFER A  \\
RULE "obviously1" IS FROM A INFER B  \\
RULE "obviously2" IS FROM A AND B INFER C  \\
RULE "obviously3" IS FROM A AND B AND C INFER D  \\
RULE "obviously4" IS FROM A AND B AND C AND D INFER E 
\end{japeish}
and the tactic which applies them is
\begin{japeish}
TACTIC obviouslytac IS \\
\tab WHEN \\
\tab (LETHYPS \_A /* 1 or more selections, as a tuple */ \\
\tab \tab (LETLISTMATCH \_A1 \_B \_A /* \_A=(\_A1, ...), \_B=(...) */ \\
\tab \tab \tab (WHEN \\
\tab \tab \tab \tab (LETLISTMATCH \_B1 \_C \_B /* \_B=(\_B1,...), \_C=(...) */ \\
\tab \tab \tab \tab \tab (WHEN \\
\tab \tab \tab \tab \tab \tab (LETLISTMATCH \_C1 \_D \_C ... etc. ...)\\
\tab \tab \tab \tab \tab \tab (LAYOUT "obviously, from" ALL \\
\tab \tab \tab \tab \tab \tab \tab "obviously2" (WITHHYPSEL (hyp \_A1)) (WITHHYPSEL (hyp \_B1))))) \\
\tab \tab \tab \tab (LAYOUT "obviously, from" ALL \\
\tab \tab \tab \tab \tab "obviously1" (WITHHYPSEL (hyp \_A1)))))) \\
\tab (LAYOUT "obviously" ALL "obviously0")
\end{japeish}

\section{N-ary intro and elim steps}

If you have a conclusion which is a long conjunction, it takes lots of ∧ intro steps to break it apart, and they clutter up the proof horribly. It's reasonable to hide the internal reasoning, and provide a single step which appears to be an N-ary ∧ intro. It's all done with \textsc{layout compress}, which will take a root with label X and make an N-ary node by hiding any immediate child steps which have the same label. 

First I mess with the Backward menu inherited from \textj{I2L.jt}, to get the new entry in a prominent place:
\begin{japeish}
MENU Backward IS \\
\tab BEFOREENTRY "∧ intro" \\
\tab \tab ENTRY "∧ intro (all at once)" IS \\
\tab \tab \tab BackwardOnlyC  \\
\tab \tab \tab \tab (QUOTE (\_A∧\_B)) \\
\tab \tab \tab \tab (Noarg "∧ intro*" "∧ intro") \\
\tab \tab \tab \tab "∧ intro" \\
\tab \tab \tab \tab "A∧B" \\
\tab RENAMEENTRY "∧ intro" "∧ intro (one step)" \\
END
\end{japeish}
∧ intro* is the tactic that does the work. In essence it is
\begin{japeish}
TACTIC "∧ intro*"  IS \\
\tab WHEN \\
\tab \tab (LETGOAL (\_P∧\_Q)\\
\tab \tab \tab (LAYOUT COMPRESS "∧ intro") \\
\tab \tab \tab "∧ intro" \\
\tab \tab \tab "∧ intro*" \\
\tab \tab \tab "∧ intro*")\\
\tab \tab SKIP
\end{japeish}
The actual tactic does stuff with tacticresult (a global variable) to try to deliver the tree path that corresponds to the leftmost conjunct in the original formula. But I don't use it, because I don't use autoselect, and in any case it doesn't work at present (the correct result is delivered, but you can't use it in a \textsc{goalpath} tactic).

Similar tricks are pulled with sequences and Ntuples of commands.

Multiple compressed forward steps are harder. The menu tactic boils down to
\begin{japeish}
LETHYP \_P  (MATCH ("∧ elim*" \_P))
\end{japeish}
The tactic is 
\begin{japeish}
TACTIC "∧ elim*"(P)  IS \\
\tab WHEN \\
\tab \tab (LETMATCH (\_P∧\_Q)  P \\
\tab \tab \tab ("∧ elim* step" \_P  "∧ elim(L)" P) \\
\tab \tab \tab ("∧ elim*" \_P) \\
\tab \tab \tab ("∧ elim* step" \_Q  "∧ elim(R)" P) \\
\tab \tab \tab ("∧ elim*" \_Q)) \\
\tab \tab SKIP
\end{japeish}
--- it matches its argument with \_P∧\_Q and then works on \_P and \_Q. It doesn't worry about tree paths because it's working on left-hand formulae throughout. Its helper tactic is
\begin{japeish}
TACTIC "∧ elim* step"(P, rule, H) IS \\
\tab WHEN \\
\tab \tab (LETMATCH (\_P∧\_Q)  P \\
\tab \tab \tab (CUTIN \\
\tab \tab \tab \tab (LAYOUT HIDEROOT) \\
\tab \tab \tab \tab rule \\
\tab \tab \tab \tab (LETGOAL \_A (UNIFY \_A H) hyp))) \\
\tab \tab (CUTIN \\
\tab \tab \tab (LAYOUT "∧ elim") \\
\tab \tab \tab rule \\
\tab \tab \tab (LETGOAL \_A (UNIFY \_A H) hyp))
\end{japeish}
--- hide intermediate steps (those working on a formula that will be broken down further); label terminal steps ∧ elim.

\section{And that's it}

Apart from the Make Lemma action, which I (finally!) introduced with this encoding, there's nothing else new. The \textsc{cutin} action doesn't solve all forward proof problems, but multiple use of boxes is pretty unlikely, so it doesn't matter much. 

The next steps in Jape development, if there ever are any, might be to build a proper box-and-line calculator --- or they might be to devise a framework for building Jape-like devices, so that there can be a modal logic calculator, and a separation logic calculator, and I can hardly imagine what else.

