% !TEX root = roll_your_own.tex
\chapter*{Preface}

Jape is a lightweight, uncommitted, transparent proof calculator. It's designed to present an excellent graphical interface and a very short and shallow `learning curve' to all its users, whether novices learning how to make formal proofs or experts --- logicians, teachers, sofware engineers, practitioners of any kind --- describing inference systems. This manual is directed at people who have experimented with one or more of the inference systems distributed with Jape and now want to develop something of their own, or those who just want to understand what it is that Bernard Sufrin and I have done in our own encodings.


The chapters of this manual describe by example how to encode several interesting logics in Jape. They are intended to be read in sequence, as the earlier chapters give most description of the early stages of the encoding, and later chapters concentrate on more esoteric features.


A manual which described a task only by example would be inadequate, and I therefore include a complete description of the various internal `languages' of Jape:

\begin{itemize}
\item the \emph{term} or \emph{formula} or \emph{sequent} language, in which problems are stated, and which appears on-screen when a proof is displayed --- described in \appxref{paraformlang};
\item the \emph{tactic} language, which includes the statement of the inference rules of a logic, and which allows the user to control the course of a proof --- described in \appxref{tacticlang};
\item the \emph{paragraph} language, which is the notation used to describe a logic and its associated tactics and stuff to Jape --- described in \appxref{paraformlang};
\item the \emph{dialogue} language, which is the notation in which the graphical interface sends commands to the main proof engine, and in which you can type as text in a graphical interface window --- described in \appxref{GUIlang}.
\end{itemize}

This manual doesn't discuss how to use Jape --- that's covered in other manuals. The graphical illustrations are taken from the MacOS X implementation (version 7\_d4 or later).

\section*{Maintenance only}

Jape was once several experiments in one. One was an experiment in user interfaces: could we make a really nice proof calculator? Another was an experiment in abstraction: could we make it customisable? Another was educational: could we teach logic with it? And no doubt there were others which I've now forgotten.

The Jape idea isn't dead, but I think it's safe to say that Bernard and I have found other things to do with our time, and the project is stalled. It's open source, so it could conceivably get started under other hands, but I made it so damn complicated that I don't expect that will happen. Jape is now maintenance-only, though if anybody has a really good simple idea to improve it I might get itchy again.

Because of its architecture and detail internal design, some things which you might think really easy to do are in fact rather hard. So: no apologies for the following list of deficiencies, none of which are likely to be fixed.

\begin{itemize}
\item Jape doesn't check the proof store when you redefine a rule or theorem, and re-run all the proofs that depend on it (though it does guard against circularities in proofs).
\item Jape can't handle sequents in which one side is an optional single formula.
\item Jape has no treatment of definitional equality (syntactic equivalence), so you have to handle it with rules and inference steps.
\item Jape has a long-standing problem in that it can't encode `families of rules'. Bernard employed considerable ingenuity to allow you to encode finite collections of slightly different rules, but that's the nearest you'll get.
\item Jape's parser generator is horribly complicated, but awfully limited.
\item I never quite reached modal logic, and now I never will.
\item Separation logic is right out of the question.
\end{itemize}

I'll add to this list as further deficiencies are pointed out. Mail to bugs@jape.org.uk, please.

\section*{Inadequacies}

Jape started out as a Good Idea which was patched and modified to do more than it perhaps should do. I wrote this manual long after much of the work was done. You may find places in the text where I simply say that I can't remember something or other, or you may find that there are features of Jape which are used in example code but not explained in this manual. Please notify me via bugs@jape.org.uk whenever you notice anything like that.